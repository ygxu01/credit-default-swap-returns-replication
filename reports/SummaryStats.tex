\documentclass{article}

% Ensure necessary packages are included
\usepackage[letterpaper,top=2cm,bottom=2cm,left=3cm,right=3cm,marginparwidth=1.75cm]{geometry}
\usepackage{amsmath}
\usepackage{graphicx}
\usepackage{booktabs}  % For professional-looking tables
\usepackage{hyperref}
\usepackage{caption}
\usepackage{float}  % For forcing figure/table placement
\usepackage{subcaption} % For multiple figures

\title{Project 8: Summary Statistics of CDS Portfolios}
\author{Sania Zeb \& Yangge Xu}

\begin{document}

\maketitle

\begin{abstract}
This document presents figures and tables summarizing the data used in constructing the CDS portfolios. In our dataset, we identified a "sector" variable within Markit data that warrants investigation. 

This analysis provides a sectoral breakdown of CDS contracts, highlighting sector-specific trends and return distributions. These insights are valuable for understanding credit risk variations across industries and their impact on portfolio construction.
\end{abstract}

\section{Statistics of CDS Contracts Across Sectors}

\subsection{Sectoral Analysis of CDS Returns}

The table below presents the summary statistics of monthly CDS returns across different industry sectors. It includes key descriptive statistics such as mean, standard deviation, minimum, maximum, and quartiles.

\paragraph{Key Findings:}
\begin{itemize}
    \item \textbf{Financial Sector:} This sector exhibits the highest return volatility, reflecting its exposure to systemic risk.
    \item \textbf{Consumer Services:} Significant fluctuations, potentially driven by macroeconomic cycles.
    \item \textbf{Government and Healthcare:} These sectors demonstrate lower volatility, indicating relative stability in credit markets.
    \item \textbf{Mean Returns:} The average returns vary across industries, with some sectors displaying consistently negative returns, potentially indicating persistent credit risk concerns.
\end{itemize}

\subsection{Monthly CDS Returns Over Time}

\begin{figure}[H]
    \centering
    \caption{Monthly CDS Portfolio Returns Over Time. }
    \includegraphics[width=0.8\textwidth]{../_output/monthly_returns_over_time.png}  
    \caption*{
    This figure illustrates the trend of average monthly CDS returns, highlighting fluctuations that align with financial stress periods and stability phases.}
    \label{fig:monthly_cds_returns}
\end{figure}

\paragraph{Observations:}
\begin{itemize}
    \item \textbf{Overall Trend:} CDS returns remain close to zero, indicating limited long-term directional bias.
    \item \textbf{Periods of High Volatility:} Market-wide disruptions, such as financial crises, are reflected in sharp return deviations.
    \item \textbf{Post-2010 Stability:} A reduction in extreme negative returns suggests improving credit conditions.
    \item \textbf{Macroeconomic Impact:} Notable dips align with economic downturns, reinforcing the link between CDS returns and systemic risk.
\end{itemize}

\subsection{Distribution of CDS Returns by Sector}

To further investigate sector-specific risk variations, we present the boxplot of monthly CDS returns by sector. This visualization provides insights into the spread, median, and presence of extreme values.

\begin{figure}[H]
    \centering
    \caption{Distribution of Monthly CDS Returns by Sector. }
    \includegraphics[width=0.8\textwidth]{../_output/boxplot_cds_returns_by_sector.png}
    \caption*{
    The boxplot highlights return distributions across sectors, with financials exhibiting higher variance, while government and healthcare sectors demonstrate relative stability.}
    \label{fig:boxplot_cds_returns}
\end{figure}

\subsection{Sector-wise CDS Return Statistics}

\begin{table}[H]
    \centering
    \caption{Summary Statistics of Monthly CDS Returns by Sector}
    \input{../_output/latex_cds_by_sector_stats.tex}
    \caption*{
        This table presents descriptive statistics (mean, standard deviation, min, max, and quartiles) 
        for CDS returns across different industry sectors. It highlights sectoral differences in credit risk, 
        with financials and consumer services experiencing greater return variability compared to government and healthcare.}
    \label{table:latex_cds_by_sector_stats}
\end{table}

\end{document}