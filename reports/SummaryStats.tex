\documentclass{article}
%\usepackage[english]{babel}
\usepackage[letterpaper,top=2cm,bottom=2cm,left=3cm,right=3cm,marginparwidth=1.75cm]{geometry}

% Ensure necessary packages are included
\usepackage{amsmath}
\usepackage{graphicx}
\usepackage{booktabs}  % Added for tabular rules
\usepackage{hyperref}
\usepackage{caption}

\title{Project 8: Summary Statistics of CDS Portfolios}
\author{Sania Zeb \& Yangge Xu}

\begin{document}
\maketitle

\begin{abstract}
This file aims to include figures and tables that introduce and summarize the 
data used to construct the CDS portfolios in this project.

While processing the data, we found a "sector" variable in Markit data worth investigating.
We include a figure and a table that show the distribution of CDS contracts across different sectors. 
This will help us understand the sectoral composition of our dataset and its potential impact on portfolio construction and return calculation.
\end{abstract}

\section{Statistics of CDS Contracts Across Sectors}

\subsection{Sector-wise CDS Return Statistics}

\begin{table}
    \centering
    \caption{Summary Statistics of Monthly CDS Returns by Sector}
    \label{table:latex_cds_by_sector_stats}
    
    % Insert the external table file
    \input{../_output/latex_cds_by_sector_stats.tex}

    % Additional explanation as a footnote instead of another caption
    \caption*{
        This table presents descriptive statistics (mean, standard deviation, min, max, and quartiles) 
        for CDS returns across different industry sectors. The variation in return volatility highlights sectoral differences, 
        with financials and consumer services exhibiting larger fluctuations than government and healthcare.
    }
\end{table}



\subsection{Monthly CDS Returns Over Time}
\begin{figure}[h]
    \centering
    \includegraphics[width=0.8\textwidth]{../_output/monthly_returns_over_time.png}  % Adjusted path
    \caption{\label{fig:monthly_cds_returns} Monthly CDS Portfolio Returns Over Time. 
This figure illustrates the trend of average monthly CDS returns. 
Fluctuations highlight financial stress and stability periods, providing insight into the risk and performance of credit default swaps.}
\end{figure}

\end{document}
